% Generated by GrindEQ Word-to-LaTeX 
\documentclass{article} % use \documentstyle for old LaTeX compilers

\usepackage[utf8]{inputenc} % 'cp1252'-Western, 'cp1251'-Cyrillic, etc.
\usepackage[english]{babel} % 'french', 'german', 'spanish', 'danish', etc.
\usepackage{amsmath}
\usepackage{amssymb}
\usepackage{txfonts}
\usepackage{mathdots}
\usepackage[classicReIm]{kpfonts}
\usepackage{graphicx}

% You can include more LaTeX packages here 


\begin{document}

%\selectlanguage{english} % remove comment delimiter ('%') and select language if required


\noindent An\'{a}lisis estad\'{i}stico de las tres bebidas

\noindent Carlos Sebasti\'{a}n P\'{e}rez Quintana

\noindent Septiembre 2022

\noindent 

\noindent 


{\bf  Introducci\'{o}n }

\noindent El siguiente informe se encontrar\'{a} el an\'{a}lisis descriptivo de tres bebidas aleatorias, por medio de diferentes herramientas estad\'{i}sticas, en donde se involucra una media, as\'{i} como la diferencia entre dos medias, una proporci\'{o}n y la diferencia entre dos proporciones, todo este an\'{a}lisis se hace por medio de la herramienta de R-Studio.  

\noindent 

\noindent En un primer momento se tiene una data con tres elementos que son el caf\'{e}, chocolate y el agua, donde se busca medir el nivel de consumo de estas tres bebidas a una poblaci\'{o}n de 120 personas de manera aleatoria acerca sobre sus prefer\'{i}as de estas tres bebidas, para posteriormente proceder a realizar el an\'{a}lisis de prueba de hip\'{o}tesis.

\noindent 

\noindent Adicionalmente, este documento se realiza con el objetivo de identificar los gustos de las personas en cuanto a las bebidas e identificar si, de escala de uno a cinco, existe, en promedio, la misma cantidad de personas que prefieran una u otra bebida, para ello se realizara aquellas pruebas de hip\'{o}tesis mencionadas con anterioridad. 

\noindent 

\noindent En este caso para las tres bebidas tenemos las tablas de distribuci\'{o}n frecuencias para caf\'{e}, chocolate y agua.

\noindent Considerando la informaci\'{o}n anterior, es necesario informar que, se proceder a inicialmente a mostrar las diferentes tablas de distribuci\'{o}n de frecuencia, en las cuales presenta la agrupaci\'{o}n de datos en los grupos de valoraci\'{o}n excluyente, que indica la cantidad de personas que votaron por el nivel de preferencia que posee de las tres bebidas, el cual se puede realizar una comparaci\'{o}n respecto a las representaciones graficas presentadas en las figuras. 

\noindent 

\noindent 

\noindent 

\noindent 

\noindent 

\noindent 

\noindent \includegraphics*[width=5.30in, height=2.10in]{image1}

\noindent 

\noindent 

\noindent 

\noindent 

\noindent 

\noindent 

\noindent Figura 1: Tabla de distribuci\'{o}n de frecuencias.

\noindent 

\noindent Por ello, esta distribuci\'{o}n, se puede relacionar la gr\'{a}fica de la tabla del caf\'{e} y observar que los datos con mayor claridad, y cantidad de personas que votaron tiene un nivel de aceptaci\'{o}n alto o medio y donde se denota una agrupaci\'{o}n netamente excluyente, es decir que una misma persona no pudo votar por dos niveles diferentes. 

\noindent 

\noindent \includegraphics*[width=6.17in, height=4.52in]{image2}

\noindent Figura 2: Diagrama de cajas comparativo, en tres las tres bebidas, histograma, diagrama de barras 

\noindent 

\noindent Para su an\'{a}lisis es necesario tener en cuenta las diferentes representaciones ocupadas, para poder denotar que en el diagrama de barras que una gran cantidad de personas indicaron su gusto por el caf\'{e}, y esto lo podemos corroborar con el histograma, ya que indica que el de las 120 personas encuestadas 32 personas se encuentra en un nivel de gusto del 3, y el nivel de gusto 4 hay 27 personas y por \'{u}ltimo el nivel de gusto 5 se encuentra la media personas que tiene un gusto por el caf\'{e} que son 31.  Es decir que del 38.4\% de personas tienen un gusto por el caf\'{e} de nivel 3, que son 32 personas y un 37.2 tiene un gusto de 5 de 31 personas. 

\noindent Adicionalmente, si bien no podemos dar un valor especifico de la media, mediana y la moda solo con el histograma, si podemos indicar basados en el sesgo encontrado gracia al histograma que la moda es menor a la mediana, y esta a su vez menor que la media (Devore), como se demuestra en la figura 2.

\noindent Por tal raz\'{o}n, esto puede tener una mayor verificaci\'{o}n por medio del an\'{a}lisis del diagrama de cajas junto con aquellos valores obtenidos por el coeficiente de simetr\'{i}a y el c\'{a}lculo de la media, adicionalmente, para comprender de mejor manera el comportamiento de la distribuci\'{o}n, se analizar a el tipo de curtosis que describe la distribuci\'{o}n de los datos

\noindent 

\noindent \includegraphics*[width=6.12in, height=2.18in]{image3}

\noindent Figura 3: Comportamiento del sesgo de asimetr\'{i}a y curtosis 

\noindent 

\noindent Donde tiene una moda, mediana y media tanto para la derecha como para izquierda del 50\%.

\noindent La interpretaci\'{o}n de la curtosis a la curtosis como la descripci\'{o}n de que tan apuntada o achatada se encuentra una distribuci\'{o}n respecto a un comportamiento normal. Es decir que, si los datos est\'{a}n muy concentrados hacia la media, la distribuci\'{o}n es leptocurtica, si est\'{a}n muy dispersos, la distribuci\'{o}n es platicurtica y en el caso de poseer un comportamiento normal, es mesocurtica.

\noindent Ya para la interpretaci\'{o}n del diagrama de caja en la figura 1.  Se puede observar que en comparaci\'{o}n con las tres bebidas:  Primero se observa que el valor m\'{i}nimo del nivel de agrado o valoraci\'{o}n de gusto por el caf\'{e} es 1 y es el 19.2\% de 16 personas que siente gusto por el caf\'{e}, mientras que el m\'{a}ximo es 38.4\% de 32 personas. Adicionalmente se puede mencionar que el 37.2\% de 31 personas que siente un gusto de nivel 5 por el caf\'{e} las calificaciones est\'{a}n por debajo de uno, el 50 \% de las calificaciones est\'{a}n por encima de 3 a 5, es decir es donde se concentra la mayor parte de la poblaci\'{o}n que tiene el gusto por el caf\'{e}, en ese sentido si tomamos las dos poblaciones m\'{a}s gusto tendr\'{i}amos el 75.6\% de gusto en los niveles 3 y 5, y donde el nivel 4 es nuestra mediana. Otro elemento que nos indica que no hay simetr\'{i}a; el 75 \% de las calificaciones son inferior a cuatro y la totalidad de evaluaciones son inferiores a cinco o igual y en ese caso es platicurtica con un sesgo hacia la derecha.

\noindent \includegraphics*[width=2.98in, height=0.41in]{image4}

\noindent 
\[\hat{x}=\frac{\sum^n_{i=1}{x_i}}{N}=\frac{(\left(1\bullet 16\right)+\left(2\bullet 14\right)+\left(3\bullet 32\right)+\left(4\bullet 27\right)+(5\bullet 31)}{120}=0.3583\] 
En ese sentido el valor de la media esperada por R si da, pero en el caso del sesgo queda confusa, pues aparentara poseer un sesgo negativo o izquierdo, algo que con el histograma y los datos como el coeficiente de simetr\'{i}a no ser\'{i}a as\'{i}.

\noindent 

\noindent Observamos que este procedimiento nos muestra el valor del estad\'{i}stico t, sus grados de libertad'' df2 el valor p-valor del contraste 2.2e-16, esto nos indica que la hip\'{o}tesis que planteamos tiene una estimaci\'{o}n de la media de las personas que le gustan el caf\'{e} es de 3.35833 tambi\'{e}n tenemos un intervalo de confianza de 95 \%

\noindent \includegraphics*[width=4.82in, height=1.69in]{image5}

\noindent 

\noindent  Ya para finiquitar con esta bebida, hablaremos la varianza ($\alphaup$ 2), la desviaci\'{o}n este andar ($\alphaup$) y el coeficiente variaci\'{o}n (Cv), el cual este \'{u}ltimo nos refiere en cierta medida la dispersi\'{o}n de los datos. Para ello, se debe comprender que la varianza representada por $\alphaup$ 2 es la media aritm\'{e}tica de los cuadrados de las diferencias de los datos con su media aritm\'{e}tica, mientras que la desviaci\'{o}n est\'{a}ndar, la cual est\'{a} representada como $\alphaup$ es una medida de dispersi\'{o}n de los datos, en donde mayor sea la dispersi\'{o}n mayor es la desviaci\'{o}n est\'{a}ndar. En cuanto al coeficiente de dispersi\'{o}n o coeficiente de variaci\'{o}n, el cual representaremos como cv, es una medida de dispersi\'{o}n que permite el an\'{a}lisis de las desviaciones de los datos con respecto a la media y al mismo tiempo las dispersiones que tienen los datos dispersos entre s\'{i}, en donde en este caso se trabajar de manera porcentual y se realiz\'{o} algunas categor\'{i}as para definir dicha dispersi\'{o}n, las cuales se mantendr\'{a}n entre bebida a bebida. 

\noindent Dichas categor\'{i}as son: 3 

\noindent  Si el coeficiente de variaci\'{o}n esta entre el rango de 0 \% a 20 \%, este posee muy poca dispersi\'{o}n 

\noindent  Si el coeficiente de variaci\'{o}n esta entre el rango de 20 \% a 40 \%, este posee poca dispersi\'{o}n 

\noindent  Si el coeficiente de variaci\'{o}n esta entre el rango de 40 \% a 60 \%, este posee una dispersi\'{o}n moderada. 

\noindent  Si el coeficiente de variaci\'{o}n esta entre el rango de 60 \% a 80 \%, este posee una alta dispersi\'{o}n.   Si el coeficiente de variaci\'{o}n esta entre el rango de 80 \% a 100 \%, este posee una muy alta dispersi\'{o}n. 

\noindent Ya con aquella breve explicaci\'{o}n, se procede a evaluar la varianza.

\noindent 

\noindent \includegraphics*[width=3.52in, height=1.95in]{image6}  \includegraphics*[width=3.87in, height=0.61in]{image7}

\noindent 
\[{\mathrm{\alphaup }\mathrm{\ }}^2=\frac{\sum^n_{i=1}{{(x}_i\hat{x})^2}}{N}=\frac{(16(1-0.532)+(14(2-0.532)^2)(32(3-0.532)^2)(27(4-0.532)^2)(5(31-0.532)^2)}{120}=7.3709.95\] 


\noindent En otras palabras, la medida de dispersi\'{o}n de los datos es de 1.93. Ahora, con el fin de hablar del coeficiente de varianza, llegamos a la siguiente igualdad en donde:
\[C_v=\frac{\propto }{\hat{x}}=\frac{7.37}{0.532}\bullet 100=Dispersi\textrm{\'{o}}n\ moderada\ Entre\ el\ 40\%\ al\ 60\%\] 
Primeramente, podemos indicar que la media del caf\'{e} de manera redondeada es 7.37, es decir que la mitad de los datos indican que el agrado por el caf\'{e} es 7.37, indicando en cierta manera que una gran cantidad de personas no le agrada esta bebida, algo que podemos contrastar con el agua, la cual su media es 3.97, es decir muy cercana a la mediana que se hab\'{i}a calculado anteriormente, indicando en cierta manera que dentro de las tres bebidas, el agua es la bebida que m\'{a}s gusta de las tres, pues en contraste la media del capuchino es  muy cercana a la del caf\'{e}.

\noindent 

\noindent \includegraphics*[width=5.25in, height=2.08in]{image8}

\noindent 

\noindent 

\noindent 

\noindent 

\noindent 

\noindent Figura 3: Chocolate

\noindent 

\noindent Teniendo en cuenta la informaci\'{o}n anterior, permite por medio de diversos par\'{a}metros el analizar algunas graficas pre entes en la figura 3, sea de manera total o parcial respecto a la informaci\'{o}n que se presenta, una de estos es  respecto a la gr\'{a}fica de barra y la informaci\'{o}n  que refiere a la cantidad de personas que valoraron al chocolate con un nivel 1 o de disgusto que fueron 15 a diferencia del caf\'{e} que fueron 16 personas, en donde si bien no se compara con la bebida caf\'{e}, ciertamente se denota una gran cantidad, que si se analiza el diagrama  podemos entrever que entre las 120 personas entrevistadas, el 32.4\% por ciento de estas les disgusta esta bebida, que equivaldr\'{i}a seg\'{u}n el histograma aproximadamente a 27 personas en los niveles 1 y 2, pero para verificarlo podemos dirigirnos a la tabla de distribuci\'{o}n de frecuencia en donde indica que aquellas personas que evaluaron a 1 en la encuesta, consisti\'{o} de 15 que ser\'{i}a el 18\% de la poblaci\'{o}n les disgusta el chocolate, en el nivel 2 el 14.4\% . Pero como bien es sabido, el histograma y el diagrama de caja proporciona una gran informaci\'{o}n respecto al comportamiento de los datos y una comparaci\'{o}n de valores de manera general entre la media y la mediana.

\noindent Sin olvidar el gusto que tuvo la gente al chocolate que encuentra en nivel 5 con 42 personas que fue del 50.4\% y donde su mediana se ubic\'{o} con 26 personas en el nivel 3 con un 31.2\%.

\noindent 

\noindent 

\noindent \includegraphics*[width=6.06in, height=3.84in]{image9}Figura 4

\noindent 

\noindent En cuanto al diagrama de caja, se puede evidenciar que, en el primer cuartil, el cual corresponde al 25 \%, nos indica que el 25 \% de las evaluaciones de gusto del chocolate est\'{a} por debajo de 1.75, el cual fue un valor adquirido en r, esto es debido a la b\'{u}squeda de una aproximaci\'{o}n certera, m\'{a}s que un intento de describir el primer cuartil, el cual no fue necesario gracias a la figura 5

\noindent 

\noindent \includegraphics*[width=5.33in, height=1.40in]{image10}

\noindent 

\noindent Observamos que este procedimiento nos muestra el valor del estad\'{i}stico t, sus grados de libertad df2 el valor p-valor del contraste 2.2e-16, esto nos indica que la hip\'{o}tesis que planteamos tiene una estimaci\'{o}n de la media de las personas que le gustan el chocolate es de 3.3583 tambi\'{e}n tenemos un intervalo de confianza de 95 \%

\noindent \includegraphics*[width=4.95in, height=1.39in]{image11}

\noindent 

\noindent Para el agua, le gusto que tiene la poblaci\'{o}n de 120 personas:

\noindent \includegraphics*[width=5.36in, height=2.15in]{image12}

\noindent \textbf{}

\noindent \textbf{}

\noindent \textbf{}

\noindent \textbf{}

\noindent \textbf{}

\noindent \textbf{}

\noindent Figura 5

\noindent En este caso se observa que la poblaci\'{o}n de 120 personas se vio m\'{a}s conectada con el agua es decir sus niveles se encuentran en 3,4 y 5 donde concentra la mayor parte las encuestas y donde el nivel 5 de satisfacci\'{o}n se encuentra con 49 personas con el 58.8\% de gusto, en ese sentido de las tres bebidas se puede decir que la m\'{a}s gustada y bebida de las tres.  Y en total de los tres niveles de satisfacci\'{o}n para el agua 110 personas tiene un gusto intermedio y alto del agua.

\noindent \textbf{}

\noindent \textbf{}

\noindent Observamos que este procedimiento nos muestra el valor del estad\'{i}stico t, sus grados de libertad df2 el valor p-valor del contraste 2.2e-16, esto nos indica que la hip\'{o}tesis que planteamos tiene una estimaci\'{o}n de la media de las personas que le gustan el agua es de 3.966 tambi\'{e}n tenemos un intervalo de confianza de 95 \%

\noindent \includegraphics*[width=4.80in, height=1.34in]{image13}

\noindent 

\noindent \includegraphics*[width=6.10in, height=3.77in]{image14}

\noindent \textbf{}

\noindent \textbf{Muestras peque\~{n}as general:}

\noindent \textbf{}

\noindent Para esto se utiliza las'' muestras peque\~{n}as'' que nos dio los test anteriores, la proporci\'{o}n de las distintas cantidades. Donde se usa una prueba binomial para comparar una proporci\'{o}n de los 120 datos Tenemos que se hicieron 26 n\'{u}meros de \'{e}xitos y los intentos que se hicieron fueron 150 con un valor P de 0.02657 hip\'{o}tesis alternativa: la verdadera probabilidad de \'{e}xito no es igual 8 a 0.3 con un intervalo de confianza de 0.95 entonces la probabilidad de \'{e}xito es de 0.375 esto es para el caf\'{e}\textbf{}

\noindent \textbf{}

\noindent \includegraphics*[width=6.11in, height=1.96in]{image15}\textbf{}

\noindent Igualmente, para el chocolate y el agua se llev\'{o} acabo el test de muestras peque\~{n}as:

\noindent Donde sus valores cambian y muestra que para chocolate solo se tomaron 31 datos se tuvo esos 31 \'{e}xitos de los 120 y caso de \'{e}xito es de 0.25833, para el agua se toma 16 casos y se tiene 0.1333 casos de \'{e}xito.

\noindent \includegraphics*[width=6.12in, height=2.91in]{image16}

\noindent \includegraphics*[width=6.09in, height=2.66in]{image17} 

\noindent \textbf{}

\noindent \textbf{}

\noindent 

\noindent 

\noindent \textbf{}

\noindent \textbf{}

\noindent \textbf{}

\noindent \textbf{}

\noindent 

\noindent \textbf{}

\noindent \textbf{}

\noindent \textbf{}

\noindent \textbf{}

\noindent \textbf{}

\noindent \textbf{}

\noindent \textbf{}

\noindent \textbf{}

\noindent \textbf{}


{\bf  An\'{a}lisis de datos}

\noindent Para analizar los elementos a considerar, es necesaria la compresi\'{o}n estad\'{i}stica, que ayuda poder interpretar los diferentes resultados.

\noindent \textbf{}


{\bf     La media}

\noindent Se tiene la media donde se tomaron 120 resultados de las personas que beben caf\'{e}, donde cada persona califica que tanto le gusta la bebida donde el menor n\'{u}mero entre 1 y 5, es que no le gusto y el mayor n\'{u}mero es que le gust\'{o} mucho.
\[x=\frac{x_1+x_2\pm \dots +x_n}{n}\] 


{\bf   Error t\'{i}pico}

\noindent 
{\bf }

\noindent 
{\bf El promedio sigue siendo una aproximaci\'{o}n, por lo que hay que calcular el intervalo donde m\'{a}s o menos se encuentra el general. La f\'{o}rmula para calcular el Error estimado para la media en el muestreo aleatorio simple}

\noindent \textbf{}

\noindent \textbf{}

\noindent 

\noindent 

\noindent  

\noindent 

\noindent 

\noindent 

\noindent 


\end{document}

